\newpage\section{Podział i harmonogram pracy}
W rozdziale przedstawiony został podział pracy między autorami projektu \\(Tabela 1). Harmonogram pracy widoczny jest w Tabeli 2.

\begin{table}[!ht]
	\centering
	\begin{tabular}{|c|p{7cm}|}
		\hline \textit{Autor} & \textit{Podzadanie} \\ \hline
		Piotr Maleszczuk & GUI, wizualizacja ruchów, obsługa różnych trybów gry, implementacja metody Minimax \\ \hline
	
		Łukasz Błaszczak & Przygotowanie planszy, obsługa kliknięć ekranu, przeliczanie punktów, imitacja kliknięć \\ \hline
	
		Przemysław Burdelak & Odnajdywanie możliwych ruchów pionka, budowanie dostępnych plansz, podłączenie maszyny grającej do rozgrywki \\ \hline
	\end{tabular}
	\caption{Podział pracy}
\end{table}

\begin{table}[!ht]
	\centering
	\begin{tabular}{|c|p{7cm}|}
		\hline
		\textit{Lp.} & \textit{Opis zadań} \\ \hline
		1. & Przygotowanie zdalnego repozytorium \\ \hline
		2. & Przygotowanie środowiska programistycznego \\ \hline
		3. & Przygotowanie środowiska Unity oraz założenie wstępnego projektu \\ \hline
		4. & Rozpisanie projektu na Trello, podzielenie na poszczególne zadania wykonywane przez cały czas projektu. \\ \hline
		5. & Utworzenie menu startowego gry (ekran główny, opcje, sklep) \\ \hline
		6. & Utworzenie menu wyboru trybu (singleplayer, multiplayer, online multiplayer) \\ \hline
		7. & Implementacja rozgrywki - logika gry \\ \hline
		8. & Implementacja gry jednoosobowej - SI \\ \hline
		9. & Komunikacja Bluetooth \\ \hline
		10. & Implementacja gry wieloosobowej \\ \hline
		11. & System reklam \\ \hline
		12. & System zbierania danych \\ \hline
		13. & Testy i ewentualne poprawki \\ \hline
	\end{tabular}
	\caption{Harmonogram pracy}
\end{table}

