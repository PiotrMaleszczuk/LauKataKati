\newpage\section{Architektura}
Rozdział opisuje architekturę, która została wykorzystana w projekcie oraz ogólny opis funkcjonalności zaimplementowanych w danym module.

\subsection{Model View Controller}
Wzorzec projektowy służący do organizowania struktury aplikacji posiadających graficzne interfejsy użytkownika. Zastosowanie go w projekcie umożliwia wygodny podział na kontrolery, które przejmują odpowiedzialność na zadaną funkcjonalnością wykonywaną na modelach. Widoki umożliwiają oddzielenie warstwy prezentacji od wszelkich operacji wewnętrznych.

\subsection{Kontroler planszy}
Kontroler odpowiedzialny za stworzenie oraz obsługę macierzy stanu planszy, na której opiera się cała rozgrywka.  Macierz początkowa ma postać:
$$
\left| \begin{array}{ccccccc}
1 & 1 & 1 & -1 & 2 & 2 & 2 \\
1 & 1 & 1 & 0 & 2 & 2 & 2  \\
1 & 1 & 1 & -1 & 2 & 2 & 2 \\
\end{array} \right|
$$\\
-1 - pole nieużywane \\
0 - pole puste \\
1 - pole z pionkiem gracza 1 \\
2 - pole z pionkiem gracza 2

\subsection{Kontroler tur}
Główny kontroler odpowiedzialny za przebieg całej rozgrywki.\\
Odpowiada za zmianę tur graczy, przeliczanie punktów, zakończenie gry. 

\subsection{Kontroler logiki}
Kontroler na podstawie pozycji pionka oraz macierzy planszy zwraca możliwe dla niego ruchy oraz flagę czy dostępne jest kolejne bicie w tej samej turze. 

\subsection{Kontroler wizualizacji}
Jest to kontroler obsługujący wizualizację przebiegu rozgrywki. 
Przeszukuje macierz pozycji pionków - odpowiednik wirtualnej planszy. Gdy kontroler na danej pozycji w macierzy napotka wartość '1' lub '2' ustawia na odpowiadającym jej polu pionek. Jeżeli wartość ta wynosi '0' usuwa pionek z danego pola planszy.\\
\\
Dodatkowo kontroler obsługuje funkcję wyświetlania podpowiedzi możliwych ruchów dla danego pionka po jego naciśnięciu.

\subsection{Kontroler kliknięć}
Pobiera punkt kliknięcia i szuka elementu tj. pionka, pustego pola,\\
który w danym miejscu się znajduje, a następnie przesyła go do kontolera planszy.

\subsection{Kontroler Bluetooth}
Uruchamia moduł bluetooth, zależnie od wybranego trybu jako klient lub\\
serwer. Korzysta ze specjalnego pluginu, który mapuje działanie bluetooth\\
na wbudowaną obsługę połączenia sieciowego w Unity.

\subsection{Kontroler aktora Bluetooth}
W tym kontrolerze, odbywa się główna komunikacja bluetooth pomiędzy\\
klientem a serwerem. Komuinikaty wysyłane są w prostej czterocyfrowej postaci:\\
s t x y \\
,gdzie: \\
s – status komunikatu 1-setup, 2-waiting, 3-transfer \\
t – tura w rozgrywce \\
x – pozycja x kliknięcia \\
y – pozycja y kliknięcia \\

\subsection{Kontroler SI}
Odpowiedzialny za działanie sztucznej inteligencji podczas rozgrywki jednoosobowej. Maszyna grajaca korzysta ze strategi MiniMax z Alfa-Beta cięciami.

\subsection{Kontroler zapisu danych}
Odpowiada za przechowywanie danych użytkownika tj. stan konta, kupione skórki. Dodatkowo umożliwia komunikację pomiędzy scenami.

\subsection{Kontroler gromadzenia danych}
Kontroler zapisuje do pliku informacje na tematy użytkownika: specyfikacja\\ telefonu, lokalizacja, oglądanie reklam, długość korzystania z aplikacji.

\subsection{Kontroler pauzy}
Umożliwia wykonanie pauzy podczas rozgrywki. Umożliwia wyjście do głównego menu, zrestartowanie rozgrywki oraz wznowienie aktualnej.

\subsection{Kontroler końca gry}
Wywoływany po zakończeniu rozgrywki. Uruchamia reklamy, a następnie umożliwia przejście do menu głównego.

\subsection{Kontroler menu}
Odpowiada za działanie całego menu głównego: obsługa sklepu, zmiany ustawień aplikacji, uruchamianie poszczególnych trybów gry.
