\newpage\section{Wybór technologii informatycznych}
W rodziale znajduje się opis wybranych technologii informatycznych. W kolejnych podsekcjach umieszczone zostały szczegółowe opisy technologii.

\subsection{System Latex}
Oprogramowanie przeznaczone do zautomatyzowanego składu tekstu, a także związany z nim język znaczników, służący do formatowania dokumentów tekstowych i tekstowo-graficznych.

\subsection{System Github}
Hostingowy serwis internetowy przeznaczony dla projektów programistycznych wykorzystujących system kontroli wersji Git.  Wybrany ze względu na łatwość \\w użyciu.

\subsection{System Trello}
Narzędzie wykorzystane do zarządzania zadaniami podczas tworzenia projektu. Projekt w całości został rozpisany w systemie na zadania główne, składające się następnie z mniejszych szczegółowych podzadań. Pozwoliło to na przejrzysty\\ i zaplanowany z góry sposób działania.

\subsection{Język programowania C\#}
Język C\# doskonale współpracuje ze środowiskiem Unity.

\subsection{Środowisko programistyczne Visual Studio}
Zintegrowane środowisko programistyczne firmy Microsoft. Jedno z najbardziej rozwiniętych tego typu narzędzie posiadających ze sobą wiele udogodnień. \\Pomimo swojej wielkości dobrze sprawdza się w każdego rodzaju projekcie. \\W porównaniu ze środowiskiem dostarczonym przez Unity (MonoDevelop) jest to narzędzie bardziej stabilne i wygodniejsze w użyciu.

\subsection{Środowisko Unity}
Zintegrowane środowisko firmy Unity Technologies używane do tworzenia gier wideo na PC, konsole, urządzenia mobilne.\\
Wyborem w tym przypadku kierowaliśmy się głównie na podstawie informacji z internetu oraz niewielką znajomością tego narzędzia przez jednego z członków naszego zespołu. Środowisko pomimo tego, że na pierwszy rzut oka może wydawać się bardzo skomplikowane, po krótkim zapoznaniu okazuje się bardzo przyjemne w użyciu, a do tego ogromna ilość materiałów pomocniczych znajdujących się w internecie.